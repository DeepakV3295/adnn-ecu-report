\part{Implementation}

\textit{Developing multiple benchmark ANN programs for the i.MX6 target}

\chapter{Design}
\textit{Motivate and describe the HDRNN based ANNs used to benchmark training}

\section[Artificial Neural Network Development Process]{ANN Development Process}
\begin{itemize}
	\item \textit{Toolchains and Yocto recipes}
	\item \textit{State target board exploration explored in later chapter}
\end{itemize}

\section[iMX6 Processor]{i.MX6 Processor}
\textit{Information about the processor in general - ISA, Memory, Clock \dots}

\subsection[ANN Acceleration]{i.MX6 as an ANN application target}
\textit{Describe performance optimisations possible using NEON(SIMD) \dots}

\subsection[CMSIS-NN]{Available Optimisation Frameworks : CMSIS-NN}
\textit{Describe ARM's CMSIS-NN as a framework worth adding}

\section[Handwritten Digit Recognition (HDR)]{Benchmark ANN - HDRNN}
\begin{itemize}
	\item \textit{Motiviate using HDRNN to measure training performance}
	\item \textit{Describe the architecture of the HDRNN}
\end{itemize}

MNIST database is the standard benchmark for ANN applications.
ANN heavily derived from \href[]{http://neuralnetworksanddeeplearning.com}{neuralnetworksanddeeplearning.com}.
Our primary target was to evaluate the training phase of an ANN.

\subsection[HDR-NN Training]{The Learning Algorithm}
{\color{red}
	\begin{itemize}
		\item batched SGD
	\end{itemize}
}
{\color{blue}link with Theory chapter content}

\chapter{Development}

\begin{itemize}
	\item \textit{Detail the exploration on i.MX6 based board}
	\item \textit{Describe the HDRNN implementations}
\end{itemize}

\section[iMX6 Custom Board Target]{Targetting an i.MX6 based custom board}

\subsection{Testing on Device}
\textit{Process for testing on device - flashing, benchmarking (perf)}

\section{HDRNN Implementation}
\textit{Motivate why HDRNN via MNIST dataset was chosen}

With the primary focus on training, MNIST dataset was primarily loaded in an easily readable format appropriate to the corresponding paradigms

\subsection[Python - Numpy]{Python Numpy based HDRNN}

\subsection[Tensorflow Lite]{Tensorflow Lite based HDRNN}
{\color{red}
	\begin{itemize}
		\item Keras based, easy to write
		\item TFLite build less easy, still straightforward
	\end{itemize}
}

\subsection[C]{C based HDRNN}

\subsection[CPP - Eigen]{CPP based HDRNN}

\section{CMSIS-NN based Optimisations}
\textit{}

\section{General Distribution of Work}
