\part{Implementation}

\textit{Developing multiple benchmark ANN programs for the i.MX6 target}

\chapter{Design}
\textit{Motivate and describe the HDRNN based ANNs used to benchmark training}

\section[Artificial Neural Network Development Process]{ANN Development Process}
\begin{itemize}
	\item \textit{Toolchains and Yocto recipes}
	\item \textit{State target board exploration explored in later chapter}
\end{itemize}

\section[iMX6 Processor]{i.MX6 Processor}
\textit{Information about the processor in general - ISA, Memory, Clock \dots}
The i.MX6 series of ARM processors, comprising the Cortex A9 and Cortex A7, offers quad-core, dual-core and single-core configurations respectively. Each processor provides a 32/64-bit DDR3/LVDDR3/LPDDR2-800 memory interface and a number of other interfaces for connecting peripherals, such as WLAN, Bluetooth, GPS, hard drive, displays, and camera sensors


\subsection[ANN Acceleration]{i.MX6 as an ANN application target}
\textit{Describe performance optimisations possible using NEON(SIMD) \dots}

\subsection[CMSIS-NN]{Available Optimisation Frameworks : CMSIS-NN}
\textit{Describe ARM's CMSIS-NN as a framework worth adding}

\section[Handwritten Digit Recognition (HDR)]{Benchmark ANN - HDRNN}
\begin{itemize}
	\item \textit{Motiviate using HDRNN to measure training performance}
	\item \textit{Describe the architecture of the HDRNN}
\end{itemize}

MNIST database is the standard benchmark for ANN applications.
ANN heavily derived from \href[]{http://neuralnetworksanddeeplearning.com}{neuralnetworksanddeeplearning.com}.
Our primary target was to evaluate the training phase of an ANN.

\subsection[HDR-NN Training]{The Learning Algorithm}

\textit{Outline Batched Stochastic Gradient Descent using Back Propagation. Necessary Details will be layed out in Theory chapter.}

\chapter{Development}

\begin{itemize}
	\item \textit{Detail the exploration on i.MX6 based board}
	\item \textit{Describe the HDRNN implementations}
\end{itemize}

\section[iMX6 Custom Board Target]{Targetting an i.MX6 based custom board}

\subsection{Overview of the Board H/W}
The board is designed for high performance low power applications and is configured with a single Cortex A9 core. This core utilises the ARMv7 architecture and features an advanced floating point unit (FPU) for efficient computation and storage of floating point numbers. Moreover, it supports the full range of single-instruction multiple-data (SIMD) instructions, allowing for SIMD vector operation to be executed quickly. Additionally, Cortex A9 includes NEON Technology, which is an advanced media processing engine for applications such as video and image encoding and decoding. This allows for efficient processing of multimedia data and multimedia applications to be implemented with minimal performance degradation

\subsection{Testing on Device}
\textit{Process for testing on device - flashing, benchmarking (perf)}

\section{HDRNN Implementation}
\textit{Motivate why HDRNN via MNIST dataset was chosen}

With the primary focus on training, MNIST dataset was primarily loaded in an easily readable format appropriate to the corresponding paradigms

\subsection[Python - Numpy]{Python Numpy based HDRNN}

\subsection[Tensorflow Lite]{Tensorflow Lite based HDRNN}

Developing ANNs on tensorflow using Keras is straightforward with good support and well documented APIs. Building the same model for a Tensorflow Lite (TFLite) was more involved however still straightforward

\subsection[C]{C based HDRNN}

\subsection[CPP - Eigen]{CPP based HDRNN}

\section{CMSIS-NN based Optimisations}
\textit{}

\section{General Distribution of Work}
