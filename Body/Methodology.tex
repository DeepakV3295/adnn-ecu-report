\part{Implementation}

\textit{Approach to writing Tiny Neural Networks and the nature of their target environments}

\chapter{Design}
\textit{Describe the system design of the Real-Time Linux environment, support provided for the neural network execution, performance engineering based on the hardware, and design and optimisation of the neural network that is executed}

{\color{yellow}
	NOTE: Talk about the reason why MNIST database was chosen and also elaborate how the idx gunziped version was also avoided to remove the dependency of a gzip library crate / package etc
}

\section[Embedded Operating System]{Embedded Linux Environment}
\textit{Information regarding configuration and other details}

\subsection{Neural Network Support}
\textit{Leveraging hardware support for the application from the OS layer}

\subsection{CMSIS-NN Kernels}
\textit{Utilising ARM's CMSIS-NN Kernels}

\subsection{Hooks for the Applications}
\textit{Application design}

\section{Tiny Neural Network}
\textit{The architecture of the Anomally Detection Neural Network}

\chapter{Development}
\textit{Details of how different neural networks were implemented}

\section{General Distribution of Work}

\section{The C Implementation}
\textit{}

\section{CMSIS-NN Implementation}
\textit{}

\section{Testing on Device}
\textit{Process for testing on device}

\subsection{Flashing the application}
\textit{Where the device comes in the development loop}

\subsection{Performance Evaluation}
\textit{Perf tools and profiling techniques}
