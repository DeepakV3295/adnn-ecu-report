\part{Introduction}

Embedded devices are a category of tiny devices with physical, computational, and real-time computing constraints constituting an embedded system. An embedded system is a combination of hardware and software components put together to achieve specific tasks. Often, embedded systems are built into a larger device or system where they are used to collect, store, process, and analyse data, as well as to control the device\textquotesingle s behaviour.

Like most of the automotive industry, Scania employs embedded devices called Electronic Control Units (ECUs) in their trucks to supervise and regulate essential subsystems such as the engine, transmission, braking, and electrical systems. Each of these subsystems has one or more ECUs to gather system data and transmit it to a central communicator via the Controller Area Network (CAN) interface. The existing fleet of trucks will receive upgraded ECUs and communicators with higher performance hardware enabling faster and advanced data collection. This upgrade presents an opportunity to reuse the on-board hardware that is currently operational.

One such opportunity lies in performing machine learning (ML) applications on these ECUs. Machine learning has the ability to provide real-time insight and intelligence to devices however their implementation in these systems presents a unique set of challenges due to limited resources available on embedded devices. Embedded systems are designed to be power efficient, have limited memory and processing power, and require closely tailored algorithms, making it difficult to use pre-existing machine learning models. Furthermore, embedded systems often have expectations such as real-time behaviour which can complicate the ML development process. Despite these challenges, machine learning on embedded systems has potential applications in a variety of areas, such as in the fields of robotics and autonomous vehicles.

One such application Scania has been working on in their \textsc{LOBSTR} \cite{lobstr} and \textsc{FAMOUS} projects is anomaly and fault detection. Targeting to run the anomaly detection models on the existing ECUs with limited resources provides many benefits.

Scania currently runs a massive fleet of more than 100,000 trucks and has been adding another 60,000 trucks annually since 2014. The company's truck sales make up 62\% of its global sales. Scania has a substantial fleet of connected trucks with electronic control units (ECUs) and communication devices that are due for an upgrade. However, this upgrade will result in a massive amount of e-waste, which could be prevented. Scania is actively working towards a sustainable and autonomous future by devising strategies to promote a shift towards eco-friendly transport systems.

As part of this commitment, Scania is exploring the possibility of repurposing existing ECU's to run machine learning models. This innovative approach aligns with Scania's vision of leading the way towards a sustainable future.

 % The company is committed to promoting a shift towards eco-friendly transport systems, which is crucial for reducing carbon emissions and creating a greener future.

% Reason :
	% 1. Sustainabililty - check
	% 2. Potential in tiny ML (billions of devices) and Federated Learning approaches
			% Needs Training on board

% Gap  in the market: No support for training ML models on embedded devices
% Scope: Exploring the challenges of building and training NN's on embedded devices using different approaches and evaluating their performances.

% TODO: Attach a reference for the following claim

Among the ML approaches to take, Artifical Neural Networks are especially interesting for anomaly detection due to several factors such as minimal data preprocessing \dots To get the best out of this approache, training of the model needs to be performed on-board. However much of the potential of running machine learning applications on these devices remain unattained due to the difficulties in creating these applications and running training on-board.

\subsubsection{Problem Description}

The scope of the thesis is to explore the challenges of building and training artificial neural networks on embedded devices using different approaches and evaluating their performance.

%An important feature of machine learning applications are their iterative improvement process. For neural network applications this happens during the training process which traditionally consumes a lot of compute resource

% \subsubsection{Why perform training and inference on ECUs?}

% In the world of embedded systems resources such as compute, memory, network bandwith etc. are all limited. The traditional model of sending data from embedded device sensors off-board to compute clusters on the cloud presents several challenges such as bandwith consumption, privacy considerations, and more that makes it attractive to perform both training and inference on-board the embedded device

% \paragraph{Federated Learning}{
% 	One approach to making this training loop take place from within these platforms is Federated Learning which cruicially allows for the data to remain on the device
% }

\chapter{Background}

Developing and maintaining applications that rely on neural network models on a fleet of embedded devices has several considerations. The application deployment process should allow for continual updates to the neural network, transfer data or model updates from the embedded devices to off-board analytics or machine learnining pipelines, and not interfere with the other applications on the embedded device while maintaining correct representations in the neural network model. It is thus important to have an operating system that can support these applications with features such as process isolation, inter process communication mechanisms, multitasking etc.

The target embedded devices to run these applications are the ECUs aboard a Scania truck which have application processor cores that are capable of running rich operating systems such as linux distributions or real-time operating systems such as QNX, or VxWorks. All these operating systems also support hypervisors which allows for configurations where a host operating system runs automotive applications in addition to a guest operating system . Linux is

The next section looks at developing such an embedded linux environment and the process of developing neural network applications for that operating system.

\section[Development Process for Embedded Linux]{Development For Embedded Linux}

Building the linux kernel requires .An overview of these build systems are presented in \hyperref[buildsystems]{Appendix I}.

\subsection[SDKs \& Compiler Toolchains]{Toolchains \& Cross compilers}

Creating applications that are to be run on an embedded devices requires a set of software components that are usually collectively referred to as a Software Development Kits (SDK). This suite usually contains a toolchain that is capable of converting C or C++ source code into executables that can be run on the target embedded device

Supporting embedded hardware requires a software stack that includes several components covered in the preceding section. The initial target machine was an ECU filling the role of a coordinator on the truck. However due to certain components missing from the stack layed out previously, namely board level support components such as Yocto meta layer, or the board level The details of the attempt at uncovering this information is layed out in \hyperref[rtc-c300]{Appendix II}. Ultimately a similiar board, namely the , with the required information publicly provided by NXP - the vendors that provide the processor chip on the intial target ECU

\section{Training on Device}

\textit{Mention traditional offboard training and onboard inference architecture vs approaches with training on board. Reference Tiny ML research}

\subsection{Federated Learning}

\textit{link to federated learning, mention FAMOUS again}

\section{Neural Network Applications on Embedded Devices}

Neural network applications are generally written using machine learning libraries such as Tensorflow or PyTorch

\subsection{Different Programming Paradigms}
\textit{Approaches to doing Machine Learning in Embedded Environments. Emphasis on how these applications are developed - e.g TFLite}

\section{General Distribution of Work}

\chapter{Theory}

\section{Neural Networks}

\section{Support for Neural Networks in Embedded Hardware}

\section{Embedded Linux Environment}
\textit{Describe the process of boot flow introducing concepts such as Boot ROM, eMMC, IVT, bootloader, kernel, file systems etc.}

\section{Performance Evaluation}

Performance evaluation of programs
