\part{Introduction}

Embedded devices are a category of tiny devices with physical, computational, and real-time computing constraints constituting an embedded system. An embedded system is a combination of hardware and software components put together to achieve specific tasks. Often, embedded systems are built into a larger device or system where they are used to collect, store, process, and analyse data, as well as to control the device\textquotesingle s behaviour.

Like most of the automotive industry, Scania employs embedded devices called Electronic Control Units (ECUs) in their trucks to supervise and regulate essential subsystems such as the engine, transmission, braking, and electrical systems. Each of these subsystems has one or more ECUs to gather system data and transmit it to a central communicator via the Controller Area Network (CAN) interface. The existing fleet of trucks will receive upgraded ECUs and communicators with higher performance hardware enabling faster and advanced data collection. This upgrade presents an opportunity to reuse the on-board hardware that is currently operational.

One such opportunity lies in performing machine learning (ML) applications on these ECUs. Machine learning has the ability to provide real-time insight and intelligence to devices however their implementation in these systems presents a unique set of challenges due to limited resources available on embedded devices. Embedded systems are designed to be power efficient, have limited memory and processing power, and require closely tailored algorithms, making it difficult to use pre-existing machine learning models. Furthermore, embedded systems often have expectations such as real-time behaviour which can complicate the ML development process. Despite these challenges, machine learning on embedded systems has potential applications in a variety of areas, such as in the fields of robotics and autonomous vehicles.

One such application Scania has been working on in their \textsc{LOBSTR} \cite{lobstr} and \textsc{FAMOUS} projects is anomaly and fault detection. Targeting to run the anomaly detection models on the existing ECUs with limited resources provides many benefits.

Scania currently runs a massive fleet of more than 100,000 trucks and has been adding another 60,000 trucks annually since 2014. The company's truck sales make up 62\% of its global sales. Scania has a substantial fleet of connected trucks with electronic control units (ECUs) and communication devices that are due for an upgrade. However, this upgrade will result in a massive amount of e-waste, which could be prevented. Scania is actively working towards a sustainable and autonomous future by devising strategies to promote a shift towards eco-friendly transport systems.

As part of this commitment, Scania is exploring the possibility of repurposing existing ECU's to run machine learning models. This innovative approach aligns with Scania's vision of leading the way towards a sustainable future.

 % The company is committed to promoting a shift towards eco-friendly transport systems, which is crucial for reducing carbon emissions and creating a greener future.

% Reason :
	% 1. Sustainabililty - check
	% 2. Potential in tiny ML (billions of devices) and Federated Learning approaches
			% Needs Training on board

% Gap  in the market: No support for training ML models on embedded devices
% Scope: Exploring the challenges of building and training NN's on embedded devices using different approaches and evaluating their performances.

% TODO: Attach a reference for the following claim

Among the ML approaches to take, Artifical Neural Networks are especially interesting for anomaly detection due to several factors such as minimal data preprocessing \dots To get the best out of this approache, training of the model needs to be performed on-board. However much of the potential of running machine learning applications on these devices remain unattained due to the difficulties in creating these applications and running training on-board.

\subsubsection{Problem Description}

The scope of the thesis is to explore the challenges of building and training artificial neural networks on embedded devices using different approaches and evaluating their performance.

%An important feature of machine learning applications are their iterative improvement process. For neural network applications this happens during the training process which traditionally consumes a lot of compute resource

% \subsubsection{Why perform training and inference on ECUs?}

% In the world of embedded systems resources such as compute, memory, network bandwith etc. are all limited. The traditional model of sending data from embedded device sensors off-board to compute clusters on the cloud presents several challenges such as bandwith consumption, privacy considerations, and more that makes it attractive to perform both training and inference on-board the embedded device

% \paragraph{Federated Learning}{
% 	One approach to making this training loop take place from within these platforms is Federated Learning which cruicially allows for the data to remain on the device
% }

\chapter{Background}

The development and maintainance of neural network applications on a fleet of embedded devices has several design considerations and relies heavily on technological innovations. This makes embedded linux an attractive platform to build these applications and motivates an examination of build systems that target embedded environments.

\section[Development Process for Embedded Linux]{Development For Embedded Linux}

Build systems are . An overview is presented in Appendix \hyperref[buildsystems]{I}

\subsection[SDKs \& Compiler Toolchains]{Toolchains \& Cross compilers}
\textit{Describe their usage. Will show up again in Development chapter}

Support for embedded hardware requires a stack that includes several software components covered in this chapter. The initial target machine was an ECU filling the role of a coordinator on the truck, however due to certain components missing from are contained in the Appendix \hyperref[rtc-c300]{II}

\section{Training on Device}

\textit{Mention traditional offboard training and onboard inference architecture vs approaches with training on board. Reference Tiny ML research}

\subsection{Federated Learning}

\textit{link to federated learning, mention FAMOUS again}

\section[Development of Neural Network Application]{Development Of Neural Network Application}
\textit{Contrast general purpose frameworks - TFlite etc with handwritten applications}

\subsection{Different Programming Paradigms}
\textit{Approaches to doing Machine Learning in Embedded Environments. Emphasis on how these applications are developed - e.g TFLite}

\section{General Distribution of Work}

\chapter{Theory}

\section[Artificial Neural Network (ANN)]{Artificial Neural Networks}
\textit{General introduction to ANNs. Explaining topics from inference, training, till federated learning systems}

\section{ANN Performance Optimisations Techniques}
\textit{Contrast traditional implementations in resource rich environments and the constraints of embedded environment. Layout general strategies to acquire performance improvements with little losses to accuracy - Purning, Quantisation}

\section{Hardware Support for Neural Networks}
\textit{Introduction to ARM-NN kernels, mentioned again in Development chapter}

\section{Embedded Linux}
\textit{Describe the process of boot flow introducing concepts such as Boot ROM, eMMC, IVT, bootloader, kernel, file systems etc.}

\section{Performance Evaluation}
\textit{Describe and motivate performance measures used in the Results chapter}
