\part{Introduction}

\textit{Neural Network training and inference in Embedded Environments. NN Optimisations, Tiny ML, Embedded Environments, i.MX6}

\subsubsection{Why perform training and inference on ECUs?}
\textit{Blurb on Federated Learning}

\chapter{Background}
\textit{Describe ECU systems, Tiny ML}

\vspace{1em}
\noindent \textbf{Hypothesis}: Training and inference of (small) neural networks in embedded systems can be considerably improved compared to general purpose neural networks frameworks

\section{Anomally Detection}
\textit{Explain the problem. Introduce terminlogy that will get explained in the next chapter}

\section{Scania Embedded Systems}
\textit{Generic blurb on ECU systems, the kind of data they generate, and the potential of federated learning}

\subsection{i.MX6 Target Processor}
\textit{i.MX6 Specifications. Blurb on the constraints}

\section{Development for Embedded Linux}
\textit{Short introduction to writing applications for embedded linux}

\subsection{Yocto Project}
\textit{Outline the Yocto Project based Build environment}

\chapter{Theory}
\textit{Theoretical foundations}

\section{Expectations for the Hardware}
\textit{Layout the Architecture of the i.MX6 - ISA and specifications. Optimisation possibilities through using the SIMD etc}

\section{Neural Network Performance Optimisations Techniques}
\textit{Contrast traditional implementations in resource rich environments and the constraints of embedded environment. Layout general strategies to acquire performance improvements with little losses to accuracy e.g Purning, Quantisation}

\section{Tiny Machine Learning}
\textit{Approaches to doing Machine Learning in Embedded Environments}

\section{ARM's CMSIS-NN}
\textit{Introduction to ARM CMSIS-NN Kernels}
