\chapter{Scania C300 Communicator} \label{rtc-c300}
% The communicator contains iMX6 series processors

Scania ECU was set to be the target hardware to benchmark the training of a machine learning model. In order to be able to port the different implementation applications and it's dependencies on the Scania ECU and successfully execute the programs, certain information regarding the hardware, kernel, supported compilers and libraries is needed. Since the existing ECU is developed by an OEM, obtaining all the information and source files is not possible. It can be achieved by replacing the existing software with a custom developed embedded linux distribution, thus repurposing the custom hardware. The biggest challenge for repurposing an custom board with no information is reverse engineering to obtain the required information for flashing a custom linux kernel. The reverse engineering learnings are stated in this section.

There were a lack of development tools from the vendor. Furthermore, there was no bootloader access from starting a vanilla board with the serial console being silent when the bootloader was functional. The documentation contained little information on the device tree and no memory layout information. The boot flow information of the processor was available from NXP but no information was available on any modifications made by the system vendor. The lack of these resources made it difficult to port or install any tool/package on the device to reveal certain basic information. The information that was available about the processor were the number of cores, the instruction set architecture, the supported memory units, and the basic kernel information from the name, version, and distribution. There was visibility into the file system however, and to the presence of the bootloader code as well

% Research and experiments conducted on the Scania ECU to reverse engineering and obtain the required hardware and software information . Further, repurposing the ECU by flashing a custom operating system to benchmark the neural network applications was also attempted.

% The efforts were conducted both when the ECU started up in its usual operation and when it was booted in serial download mode.

The first naive approach taken was of flashing the MTD partitions that houses the bootloader. This was to verify that it is possible to flash the same bootloader and to then boot the board. A dump of the existing bootloader was taken and flashed in the same partition. This worked and the device booted successfully. Next, the u-boot bootloader developed from the yocto project was flashed on the bootloader MTD partition with the little device information that was available and sample by examing the board. However the broad was bricked, pressumably due to the u-boot image having incomplete or wrong device tree structure or perhaps the loading kernel failed as version mismatch between bootloader and kernel, or a checksum failure, or perhaps the size of the file was big enough to overwrite a different crucial region.

% (TODO: Using mfgtools on board, to collect information from bootloader.)
% (TODO: Using uuu tool in SDM mode to flash custom image.)
% (TODO: bootloader flashing using dd command)

% (TODO: results from varying bootloader environment parameters)
% (TODO: results from mfgtools experiment)
% (TODO: results of unbricking the board)
% (TODO: reasoning not enough memory layout infomation)


Exploring the target ECU board involved several examinations of a known state of the board. The linux kernel binaries were made via the Yocto project however there was no access to source code such as the recipes or the meta-layers themselves

The i.MX SoCs have a special boot mode named Serial Download Mode (SDM) typically accessible through boot switches. When configured into this mode, the ROM code will poll for a connection on a USB OTG port
