%
% Template for Doctoral Theses at Uppsala
% University. The template is based on
% the layout and typography used for
% dissertations in the Acta Universitatis
% Upsaliensis series
% Ver 5.2 - 2012-08-08
% Latest version available at:
%   http://ub.uu.se/thesistemplate
%
% Support: Wolmar Nyberg Akerstrom
% Thesis Production
% Uppsala University Library
% avhandling@ub.uu.se
%
%%%%%%%%%%%%%%%%%%%%%%%%%%%%%%%%%%%%%%%%%%%


\documentclass[a4paper]{UUThesisTemplate}

% Package to determine wether XeTeX is used
\usepackage{ifxetex}

\ifxetex
	% XeTeX specific packages and settings
	% Language, diacritics and hyphenation
	\usepackage[babelshorthands]{polyglossia}
	\setmainlanguage{english}
	\setotherlanguages{swedish}

	% Font settings
	\setmainfont{Times New Roman}
	\setromanfont{Times New Roman}
	\setsansfont{Arial}
	\setmonofont{Courier New}
\else
	% Plain LaTeX specific packages and settings
	% Language, diacritics and hyphenation
	% Use English and Swedish languages.
	\usepackage[swedish, english]{babel}

	% Font settings
	\usepackage{type1cm}
	\usepackage[latin1]{inputenc}
	\usepackage[T1]{fontenc}
	\usepackage{mathptmx}

	% Enable scaling of images on import
	\usepackage{graphicx}
	\usepackage{float}

	% Include algorithm pseudocode
	\usepackage{algorithm}
	\usepackage{algpseudocode}
\fi

% Diagrams
\graphicspath{ {./Body/images/} }

% Tables
\usepackage{booktabs}
\usepackage{tabularx}

% Document links and bookmarks
\usepackage{hyperref}
\usepackage{url}

% Numbering of headings down to the subsection level
\numberingdepth{subsection}

% Including headings down to the subsection level in contents
\contentsdepth{subsection}

\author{Prasanth Shaji, Deepak Venkataram}
\title{Training Neural Networks \\ on Embedded Devices}
\subtitle{Neural Network Frameworks vs Systems Programming Languages}
\titlepagelogo{UU_logo_color.svg}

\abstractthesis{
	There is great potential in enabling neural network applications in embedded devices and an important step in that is to allow for these devices to perform the training of the neural network on board the device. Neural network inference is a popular and well-supported functionality on these platforms however neural network training still has ways to go. In this project we take a closer look at this step and try to compare the performance capabilities of popular machine learning frameworks with straight forward implementation approaches. This report also contains a discussion on the nature of implementing neural network applications on top the fragmented embedded ecosystem.

	\begin{tikzpicture}[overlay, opacity = 0.9]
	\node [anchor=south west, xshift=-2cm, yshift=-22cm] at (current page.south west)
		{\includesvg[width = \linewidth, height = \linewidth]{uu-sigil/UU_sigill_5proc_down_left.svg}};
	\end{tikzpicture}
}

\begin{document}
\frontmatter

	\frontmatterThesis

	% Optional dedication
	% \dedication{dedication ...}

	\begingroup
		% To adjust the indentation in your table of contents, uncomment and enter the widest numbers for each level
		%  E.g.  \settocnumwidth{widest chapter number}{widest section number}{widest subsection number}...{...}
		%  \settocnumwidth{5}{4}{5}{3}{3}{3}
		\tableofcontents
	\endgroup

	% Optional tables
	%\listoftables
	%\listoffigures

\mainmatter
	% Body of the Report
	\part{Introduction}

Scania systems are made of a diverse multitude of Electronic Control Units (ECUs) with varying features and constraints, with newer hardware with richer capabilities set to join down the pipeline. Repurposing some of the older hardware and utilising the incoming powerful hardware for machine learning (ML) applications provide an exciting frontier.

\chapter{Background}

Tiny Machine Learning (Tiny ML) is a field of study in embedded systems and machine learning that is growing at a fast pace and has a lot of relevance to this project.

\section{Scania Embedded Systems}

Nature of the Distributed Network

\subsection[Target Hardware]{Target Hardware \linebreak[2]Raspberry Pi Zero}

The Constraints on the Environment

\section{Anomally Detection}

Explain the problem and introduce some terminology which will appear again in the next chapter \dots

\subsubsection{Why perform training and inference on ECUs?}
State the Motivation

\section{Implementations using General Frameworks}

Training and inference of (small) neural networks in embedded systems can be considerably improved compared to general purpose neural networks frameworks

    % \begin{figure}[!ht]
    % \includegraphics[width=12cm]{Example/Fonts}
	% \caption{Acrobat document properties and fonts display}
    % \end{figure}

% \vspace{\baselineskip}
% \noindent Please send any questions, comments or macro contributions to\\espik@ub.uu.se.

\chapter{Theory}
This section will elaborate and build upon all the theoretical foundations required to implement most of the methods presented in this report.

\section{Pruning}

\section{Quantisation}
	\part{Implementation}

\textit{Approach to writing Tiny Neural Networks and the nature of their target environments}

\chapter{Design}
\textit{Describe the system design of the Real-Time Linux environment, support provided for the neural network execution, performance engineering based on the hardware, and design and optimisation of the neural network that is executed}

{\color{yellow}
	NOTE: Talk about the reason why MNIST database was chosen and also elaborate how the idx gunziped version was also avoided to remove the dependency of a gzip library crate / package etc
}

\section[Embedded Operating System]{Embedded Linux Environment}
\textit{Information regarding configuration and other details}

\subsection{Neural Network Support}
\textit{Leveraging hardware support for the application from the OS layer}

\subsection{CMSIS-NN Kernels}
\textit{Utilising ARM's CMSIS-NN Kernels}

\subsection{Hooks for the Applications}
\textit{Application design}

\section{Tiny Neural Network}
\textit{The architecture of the Anomally Detection Neural Network}

\chapter{Development}
\textit{Details of how different neural networks were implemented}

\section{General Distribution of Work}

\section{The C Implementation}
\textit{}

\section{CMSIS-NN Implementation}
\textit{}

\section{Testing on Device}
\textit{Process for testing on device}

\subsection{Flashing the application}
\textit{Where the device comes in the development loop}

\subsection{Performance Evaluation}
\textit{Perf tools and profiling techniques}

	\part{Analysis}
%present the details of the experiments
%   i. Hardware and Software info
%  ii. NN implementations : challenges in developing model & build for target HW
% iii. Different performance measurements: Training time, Model accuracy, Peak 	%	   memory utilised and memory footprint
%  iv.
A hand digit recognition neural network (HDR-NN) model is implementated in C, C++ Eigen, Python Numpy and Tensorflow/Pytorch. The performance of HDR-NN training implementations was evaluated on the iMX6SDB evaluation board, which was programmed with an Embedded Linux built using The Yocto Project. To gauge the effectiveness of the models, we compared model accuracy, execution time, and peak memory usage while altering the number of layers and neurons in each layer. Furthermore, we address the obstacles encountered in developing the NN model and compiling it to operate on the target hardware.

%Ideas: 1. Coefficient of variation
%		2. optimisation flags
%		3. Measurements using different tools
%		4. Early stopping
%		5. Choosing Model parameters
%		6. Extrapolation

\chapter{Results}
The HDR-NN models implemented in all paradigms have a constant input size of 784 and output size of 10. The hidden layer sizes vary depending on the implementation:
\begin{itemize}
	\item C and C++ Eigen: 2, 4, 8, 32, 128, (32,16), and (128,16)
	\item Python-Numpy: 2, 8, 32, (32,16)
	\item Tensorflow/Pytorch:
\end{itemize}

The model accuracy, training time, and peak memory usage are measured for each implementation and all its hidden layer settings. This experiment is repeated 10 times and the final values are the average of all iterations.

\section{Accuracy}
The different implementation perform similarly in model accuracy. This is expected as the models have the exact structure and configurations.
\begin{figure}[ht]
	\centering
	\includegraphics[scale=0.37]{accuracy}
	\caption[HDR-NN Accuracy]{Comparing the accuracy of the different HDR-NN implementations.}
\end{figure}

Further, when the number of neurons in a single layer exceeds 32, the accuracy of the model is observed to decrease due to overfitting. To improve accuracy, adding another layer with 16 neurons is found to be beneficial without significantly increasing the time required for computation. In fact, for larger network sizes, it is observed to even reduce the computation time required.


\section{Execution Time}
The run times increased exponentially with the number of parameters. This is due to the fact that the amount of calculation in a fully connected network increases with the number of neurons, leading to longer training times

\begin{figure}[ht]
	\centering
	\includegraphics[scale=0.35]{exec-time}
	\caption[Execution Time vs Model Parameters]{Comparing total run time for training the different HDR-NN programs}
\end{figure}

\section{Peak Memory Usage}
Regardless of the hidden layer sizes, the peak memory utilisation remains constant for the same application regardless of the network configuration. C++ Eigen implementation has the least run time memory footprint while Python Numpy is the worse performing.

\begin{figure}[ht]
	\centering
	\includegraphics[scale=0.31]{memory-bar}
	\caption[Peak Memory Utilisation]{Peak Memory Utilized during training with different model sizes remain similar within the same implementation}
\end{figure}

\subsection[Python - Numpy]{Python Numpy based HDR-NN}

The Numpy implementation consistantly took longer duration to perform the same training cycle as compared to the C implementation

\subsection[Tensorflow Lite]{Tensorflow-Lite based HDR-NN}
\textit{Benchmark pending \dots}

\subsection[C]{C based HDR-NN}

C implementation had lower execution times and memory usage

\subsection[CPP - Eigen]{CPP based HDR-NN}
\textit{Benchmark pending \dots}

\section{CMSIS-NN based Optimisations to Training}
\textit{Further breakdown of the performance achieved from different optimisation techniques}

\subsection{Quantisation}
\textit{future: Training Network with Quantized weights}

\subsection{Pruning the Network}
\textit{future}

\section{Coefficient of variation}
A total of 10 iterations were conducted to ensure that the results remained consistent. To assess the degree of variability among the various trials, the mean and standard deviation were calculated across all runs, and their ratio was determined. This ratio indicates the level of variation between the different tests.

\chapter{Discussion}

\section{Developer Experience}

\section{Early stopping}
The training for all the implementations were executed by configuring the number of epochs as 30. This leads to the accuracy of model dropping significantly due to overfitting, which could be avoided if early stopping was implemented.
But, early stopping is not implemented as the performance would be completely different and there wouldn't be a standard setting to compare the implementations.

\chapter{Conclusion and Future Work}
\textit{What does it all mean? Where do we go from here?}


\backmatter
	% References
	% No restriction is set to the reference styles
	% Save references in References.bib
	\nocite{*} % Remove this for your own citations
	\bibliographystyle{plain}
	\bibliography{References}

	\part{Appendixes}

	\input{Appendix/C300.tex}
	\chapter{An Example Yocto Project Set up} \label{yocto-setup}

Consider a simple development scenario targetting the embedded hardware device MCIMX6Q-SDB, the primary hardware utilized for this report. The aim is to build and run an Embedded Linux distribution for the MCIMX6Q-SDB using a personal computer running Linux. Note that several of the code listings used in this discussion may break their interface, hence the emphasis is primarily on the essential concepts involved.

The primary version control system used within the Yocto Project is git. The Yocto Project has several other software packages that are required for its build system, all of which are laid out in its documentation (see \href{https://docs.yoctoproject.org/}{docs.yoctoproject.org}). Furthermore, there are minimum versions for the required packages for a given version of its release (see \href{https://wiki.yoctoproject.org/wiki/Releases}{wiki.yoctoproject.org/wiki/Releases}).

The following listing presents the typical required packages however, based on the Linux distribution of the development host the command will vary.

\begin{minted}[bgcolor=lightgray]{bash}
PACKAGE_MANAGER install bc make automake gcc gcc-c++ chrpath cpio diffstat gawk git python texinfo wget zstd lz4
\end{minted}

The above list is an incomplete example and relies on non-standard package names. Consulting the Yocto Project documentation provides a more accurate instruction as to the package names that are necessary to start with the Yocto Project on a given Linux distribution. An example install of the packages necessary on the Fedora 37 operating system as recommended by the Yocto Project documentation is given in the following code listing. Note that the listing assumes certain administrative capabilities to successfully complete its execution.

\begin{minted}{bash}
sudo dnf install gawk make wget tar bzip2 gzip python3 unzip perl patch diffutils diffstat git cpp gcc gcc-c++ glibc-devel texinfo chrpath ccache perl-Data-Dumper perl-Text-ParseWords perl-Thread-Queue perl-bignum socat python3-pexpect findutils which file cpio python python3-pip xz python3-GitPython python3-jinja2 SDL-devel rpcgen mesa-libGL-devel perl-FindBin perl-File-Compare perl-File-Copy perl-locale zstd lz4 hostname glibc-langpack-en
\end{minted}

For this section, consider the \textit{kirkstone} Long Term Support (LTS) release of the Yocto Project. To begin clone the Poky repository from \href{https://git.yoctoproject.com}{git.yoctoproject.com}

\begin{minted}{bash}
git clone https://git.yoctoproject.org/git/poky
\end{minted}

The Poky repository contains the OpenEmbedded-Core as well as the BitBake tool that is required for the build. NXP (previously \textit{freescale}) is the system maker and system vendor for the MCIMX6Q-SDB, and they provide a BSP layer in Yocto via the meta-freescale repository.

\begin{minted}{bash}
git clone https://git.yoctoproject.org/git/meta-freescale
\end{minted}

Checkout the kirkstone version of the different components by checking out the corresponding branches of the git repos. Using a different version of the Yocto Project would primarily mean checking out to a different branch or tag at this point.

\begin{minted}{bash}
git checkout kirkstone-4.0.11 # for poky
\end{minted}

\begin{minted}{bash}
git checkout kirkstone # for meta-freescale
\end{minted}

Once the correct branches have been checked out, the next step is to bring the BitBake tool into the shell environment. The Poky repository contains scripts that facilitate this step. Create a directory (assumed that \texttt{\$BUILDDIR} holds the pathname) to manage the build configuration and output files, and run the setup script inside the Poky repo.

\begin{minted}{bash}
source poky/oe-init-build-env $BUILDDIR
\end{minted}

The script will set up BitBake along with some additional tools and scripts in the current shell environment and also creates a few directories in \texttt{\$BUILDDIR} along with some configuration files. Note that after sourcing the \texttt{oe-init-build-env} script, the shell switches to the \texttt{\$BUILDDIR} directory. The first configuration file to edit will be the \texttt{\$BUILDDIR/conf/bblayers.conf}. The \texttt{bblayers.conf} file contains a list of layers to be used in the Yocto Project set up. The \texttt{bitbake-layers} utility program can be used to list and modify the entries in the \texttt{bblayers.conf} file.

\begin{minted}{bash}
bitbake-layers show-layers # list layers in the Yocto Project set up
\end{minted}

The above code listings shows \texttt{bitbake-layers} being used to list the layers of Yocto Project set up. The command should then list the \texttt{meta}, \texttt{meta-poky}, and \texttt{meta-yocto-bsp} layers. The names of the layers may change with the particular version of the Yocto Project being used. The path corresponding to those layers and a priority value attached to the layers will also be shown by the \texttt{show-layers} command of \texttt{bitbake-layers}. The priority is a means for BitBake to choose between layers for searching a recipe for particular software package, since multiple layers may have instructions for the same package. The \texttt{meta-freescale} layer that was cloned in the previous command to some path \texttt{\$META\_FREESCALE\_PATH} can then be added to the Yocto Project configuration using the same utility.

\begin{minted}{bash}
bitbake-layers add-layer $META_FREESCALE_PATH
\end{minted}

The next step is to configure the build system for the MCIMX6Q-SDB. The Note that this step may also be avoided by setting the \texttt{\$MACHINE} environment variable.

\begin{minted}{bash}
# edit inside $BUILDDIR/conf/local.conf
MACHINE = "imx6qdlsabresd"
\end{minted}

After \texttt{\$MACHINE} is configured correctly, BitBake can create a minimal image that can boot the MCIMX6Q-SDB by building the packages for \texttt{core-image-minimal} recipe. BitBake can parse a recipe file to determine a series of tasks to generate a software package or some binary artefacts. To satisfy the \texttt{core-image-minimal} recipe, BitBake has to perform a series of operations. BitBake has to parse a list of recipe files that are associated with the \texttt{core-image-minimal} recipe, examine the dependencies defined in the recipes and recursively determine all the software components that are required.

\begin{minted}{bash}
bitbake core-image-minimal # generate images for MCIMX6Q-SDB
\end{minted}

At this point the Yocto Project has been configured to find the correct metadata for building software packages and the Linux kernel for the MCIMX6Q-SDB. BitBake will proceed by first going through the \texttt{core-image-minimal} recipe to determine the order and list of tasks to be executed. After listing some information on the particular build configuration, it will then proceed to execute the tasks. The generated output images can be found in \texttt{\$BUILDDIR/tmp/deploy/images}. The path ends at the same name as the \texttt{\$MACHINE} variable name and contains several binary images. One way to use the resulting binary images to boot the MCIMX6Q-SDB is to use an SD card and the \textit{rootfs} image. The rootfs image should be named with the name of the machine, a time stamp, and the string \texttt{"rootfs"}.

\begin{figure}[h]
	\centering
	\begin{tikzpicture}
		\node (image) at (0, 0)
			{\includegraphics[scale=0.34]{MCIMX6Q-SDB-BD.png}};
		\draw[rounded corners=1ex, line width=2.1pt, yellow] (-6.5, 0.7) rectangle (-5.2, 1.9);
	\end{tikzpicture}
	\caption{MCIMX6Q-SDB-BD}
	\label{fig:mcimx6q-sdb}
\end{figure}

Booting the board using this image is one way to verify the build completed by BitBake. Obtain and flash an SD card with the \textit{rootfs} image using a suitable program such as \texttt{dd}. Furthermore, the MCIMX6Q-SDB should be configured to boot from an SD card slot. The board can be configured to boot from the SD-3 slot by toggling the dip switches (shown in yellow highlights in Figure \ref{fig:mcimx6q-sdb}) to the configuration : \texttt{D1-OFF D2-ON D3,4,5,6-OFF D7-ON D8-OFF}.

Assuming the previous steps were completed in a correct manner, the board should boot using the Embedded Linux that was created by BitBake using the \texttt{core-image-minimal} recipe. The Micro-B USB plug beside the previously mentioned dip switches can then be used to start a serial terminal emulator program using USB UART using a 115200 baud rate. Alternatively, the board should be accessible via the Ethernet interface also present on the MCIMX6Q-SDB.

	\chapter{Profiling Benchmark Applications} \label{hdrnn-profile}

Flame graphs capturing application performance during a single epoch are listed in the following section. These graphs were formed using the data obtained by the \texttt{perf} profiling utility program. \texttt{perf} was used to collect stack trace information and then this data was used to produce the flame graphs. Flame graphs are an efficient means of representing a large amount of stack trace information. The horizontal axis shows the stack profile for the different stacks that occupy the CPU during the program's execution. The vertical axis shows the stack depth, counting from zero at the bottom. Each rectangle in the graph represents a stack frame. The wider a frame is, the more often it was present in the sampling of the stack traces. The top edge shows what is on CPU and beneath it is its adjacent frames.

\subsection*{Flame graphs from perf Measurements}

\begin{figure}[H]
	\centering
	\includegraphics[scale=0.34]{c-math.h-fg.png}
	\caption{c-math.h}
\end{figure}

\begin{figure}[H]
	\centering
	\includegraphics[scale=0.34]{cpp-eigen-fg.png}
	\caption{cpp-eigen}
\end{figure}

\begin{figure}[H]
	\centering
	\includegraphics[scale=0.34]{cpp-libtorch-fg.png}
	\caption{cpp-libtorch}
\end{figure}

\begin{figure}[H]
	\centering
	\includegraphics[scale=0.34]{python-numpy-fg.png}
	\caption{python-numpy}
\end{figure}

\subsection*{Memory Profile through Heaptrack} \label{hdrnn-memory-profile}

Heaptrack measurement for C based implementation of HDR-NN running for 4 epochs under different sizes are presented below

\begin{figure}[ht]
	\centering
	\includegraphics[scale=0.15]{heaptrack-compare.png}
	\caption{\texttt{c-math.h} with different shapes. Plot on the left is for shape 8 while the right is for shape 16.}
	\label{fig:memory-shape}
\end{figure}

The memory allocations of different size as tracked by Heaptrack for a single epoch execution run of the \texttt{c-math.h} implementation for shape 8 is given below.

\begin{figure}[ht]
	\centering
	\includegraphics[scale=0.32]{c-math.h-allocations.png}
	\caption{\texttt{c-math.h} allocations}
\end{figure}


\end{document}
