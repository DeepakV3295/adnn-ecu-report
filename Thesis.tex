%
% Template for Doctoral Theses at Uppsala
% University. The template is based on
% the layout and typography used for
% dissertations in the Acta Universitatis
% Upsaliensis series
% Ver 5.2 - 2012-08-08
% Latest version available at:
%   http://ub.uu.se/thesistemplate
%
% Support: Wolmar Nyberg Akerstrom
% Thesis Production
% Uppsala University Library
% avhandling@ub.uu.se
%
%%%%%%%%%%%%%%%%%%%%%%%%%%%%%%%%%%%%%%%%%%%


\documentclass{UUThesisTemplate}

% Package to determine wether XeTeX is used
\usepackage{ifxetex}

\ifxetex
	% XeTeX specific packages and settings
	% Language, diacritics and hyphenation
	\usepackage[babelshorthands]{polyglossia}
	\setmainlanguage{english}
	\setotherlanguages{swedish}

	% Font settings
	\setmainfont{Times New Roman}
	\setromanfont{Times New Roman}
	\setsansfont{Arial}
	\setmonofont{Courier New}
\else
	% Plain LaTeX specific packages and settings
	% Language, diacritics and hyphenation
	% Use English and Swedish languages.
	\usepackage[swedish, english]{babel}

	% Font settings
	\usepackage{type1cm}
	\usepackage[latin1]{inputenc}
	\usepackage[T1]{fontenc}
	\usepackage{mathptmx}

	% Enable scaling of images on import
	\usepackage{graphicx}
\fi

% Draw diagrams
\usepackage{tikz}
\usepackage{pgfplots}
\pgfplotsset{width=10cm,compat=1.9}
\usepackage{listofitems} % for \readlist to create arrays

\tikzset{>=latex} % for LaTeX arrow head

\tikzstyle{node}=[
	very thick,circle,
	draw=Cerulean,
	minimum size=22,
	inner sep=0.5,
	outer sep=0.6
]
\tikzstyle{connect}=[->,thick,RoyalPurple,shorten >=1]

\tikzset{ % node styles, numbered for easy mapping with \nstyle
	node 1/.style={
		node,OliveGreen,
		draw=JungleGreen,
		fill=JungleGreen!25
		},
	node 2/.style={
		node,RoyalPurple,
		draw=Cerulean,
		fill=Cerulean!20
		},
	node 3/.style={
		node,BrickRed,
		draw=OrangeRed,
		fill=OrangeRed!20
		},
}

% map layer number onto 1, 2, or 3
\def\nstyle{int(\lay<\Nnodlen?min(2,\lay):3)}

% Tables
\usepackage{booktabs}
\usepackage{tabularx}

% Document links and bookmarks
\usepackage{hyperref}

% Numbering of headings down to the subsection level
\numberingdepth{subsection}

% Including headings down to the subsection level in contents
\contentsdepth{subsection}


% Uncomment to use a custom abstract dummy text
%\abstractdummy{
%	\begin{abstract}
%		Please use no more than 300 words and avoid mathematics or complex script.
%	\end{abstract}
%}

\author{Prasanth Shaji, Deepak Venkataram}
\title{Training Neural Networks on Embedded Devices}
\subtitle{Comparing Training on Neural Network Frameworks vs Systems Programming Languages like C/C++}
\titlepagelogo{UU_logo_sv_42.pdf}

\abstractdummy{Short abstract}

\begin{document}
\frontmatter
	% Creates the front matter (title page(s), abstract, list of papers)
	% for either a Comprehensive Summary or a Monograph.
	% Authors of Comprehensive Summaries use this front matter
	\frontmatterCS
	% Monograph authors use this front matter
	% \frontmatterMonograph

	% Optional dedication
	% \dedication{dedication ...}

	% Environment used to create a list of papers
	% \begin{listofpapers}
	% 	\item A Paper Discussed in this Thesis \label{apaperlabel}
	% \end{listofpapers}

	\begingroup
		% To adjust the indentation in your table of contents, uncomment and enter the widest numbers for each level
		%  E.g.  \settocnumwidth{widest chapter number}{widest section number}{widest subsection number}...{...}
		%  \settocnumwidth{5}{4}{5}{3}{3}{3}
		\tableofcontents
	\endgroup

	% Optional tables
	%\listoftables
	%\listoffigures

\mainmatter
	% Body of the Report
	\part{Introduction}

Scania systems are made of a diverse multitude of Electronic Control Units (ECUs) with varying features and constraints, with newer hardware with richer capabilities set to join down the pipeline. Repurposing some of the older hardware and utilising the incoming powerful hardware for machine learning (ML) applications provide an exciting frontier.

\chapter{Background}

Tiny Machine Learning (Tiny ML) is a field of study in embedded systems and machine learning that is growing at a fast pace and has a lot of relevance to this project.

\section{Scania Embedded Systems}

Nature of the Distributed Network

\subsection[Target Hardware]{Target Hardware \linebreak[2]Raspberry Pi Zero}

The Constraints on the Environment

\section{Anomally Detection}

Explain the problem and introduce some terminology which will appear again in the next chapter \dots

\subsubsection{Why perform training and inference on ECUs?}
State the Motivation

\section{Implementations using General Frameworks}

Training and inference of (small) neural networks in embedded systems can be considerably improved compared to general purpose neural networks frameworks

    % \begin{figure}[!ht]
    % \includegraphics[width=12cm]{Example/Fonts}
	% \caption{Acrobat document properties and fonts display}
    % \end{figure}

% \vspace{\baselineskip}
% \noindent Please send any questions, comments or macro contributions to\\espik@ub.uu.se.

\chapter{Theory}
This section will elaborate and build upon all the theoretical foundations required to implement most of the methods presented in this report.

\section{Pruning}

\section{Quantisation}
	\part{Implementation}

\textit{Approach to writing Tiny Neural Networks and the nature of their target environments}

\chapter{Design}
\textit{Describe the system design of the Real-Time Linux environment, support provided for the neural network execution, performance engineering based on the hardware, and design and optimisation of the neural network that is executed}

{\color{yellow}
	NOTE: Talk about the reason why MNIST database was chosen and also elaborate how the idx gunziped version was also avoided to remove the dependency of a gzip library crate / package etc
}

\section[Embedded Operating System]{Embedded Linux Environment}
\textit{Information regarding configuration and other details}

\subsection{Neural Network Support}
\textit{Leveraging hardware support for the application from the OS layer}

\subsection{CMSIS-NN Kernels}
\textit{Utilising ARM's CMSIS-NN Kernels}

\subsection{Hooks for the Applications}
\textit{Application design}

\section{Tiny Neural Network}
\textit{The architecture of the Anomally Detection Neural Network}

\chapter{Development}
\textit{Details of how different neural networks were implemented}

\section{General Distribution of Work}

\section{The C Implementation}
\textit{}

\section{CMSIS-NN Implementation}
\textit{}

\section{Testing on Device}
\textit{Process for testing on device}

\subsection{Flashing the application}
\textit{Where the device comes in the development loop}

\subsection{Performance Evaluation}
\textit{Perf tools and profiling techniques}

	\part{Analysis}
%present the details of the experiments
%   i. Hardware and Software info
%  ii. NN implementations : challenges in developing model & build for target HW
% iii. Different performance measurements: Training time, Model accuracy, Peak 	%	   memory utilised and memory footprint
%  iv.
A hand digit recognition neural network (HDR-NN) model is implementated in C, C++ Eigen, Python Numpy and Tensorflow/Pytorch. The performance of HDR-NN training implementations was evaluated on the iMX6SDB evaluation board, which was programmed with an Embedded Linux built using The Yocto Project. To gauge the effectiveness of the models, we compared model accuracy, execution time, and peak memory usage while altering the number of layers and neurons in each layer. Furthermore, we address the obstacles encountered in developing the NN model and compiling it to operate on the target hardware.

%Ideas: 1. Coefficient of variation
%		2. optimisation flags
%		3. Measurements using different tools
%		4. Early stopping
%		5. Choosing Model parameters
%		6. Extrapolation

\chapter{Results}
The HDR-NN models implemented in all paradigms have a constant input size of 784 and output size of 10. The hidden layer sizes vary depending on the implementation:
\begin{itemize}
	\item C and C++ Eigen: 2, 4, 8, 32, 128, (32,16), and (128,16)
	\item Python-Numpy: 2, 8, 32, (32,16)
	\item Tensorflow/Pytorch:
\end{itemize}

The model accuracy, training time, and peak memory usage are measured for each implementation and all its hidden layer settings. This experiment is repeated 10 times and the final values are the average of all iterations.

\section{Accuracy}
The different implementation perform similarly in model accuracy. This is expected as the models have the exact structure and configurations.
\begin{figure}[ht]
	\centering
	\includegraphics[scale=0.37]{accuracy}
	\caption[HDR-NN Accuracy]{Comparing the accuracy of the different HDR-NN implementations.}
\end{figure}

Further, when the number of neurons in a single layer exceeds 32, the accuracy of the model is observed to decrease due to overfitting. To improve accuracy, adding another layer with 16 neurons is found to be beneficial without significantly increasing the time required for computation. In fact, for larger network sizes, it is observed to even reduce the computation time required.


\section{Execution Time}
The run times increased exponentially with the number of parameters. This is due to the fact that the amount of calculation in a fully connected network increases with the number of neurons, leading to longer training times

\begin{figure}[ht]
	\centering
	\includegraphics[scale=0.35]{exec-time}
	\caption[Execution Time vs Model Parameters]{Comparing total run time for training the different HDR-NN programs}
\end{figure}

\section{Peak Memory Usage}
Regardless of the hidden layer sizes, the peak memory utilisation remains constant for the same application regardless of the network configuration. C++ Eigen implementation has the least run time memory footprint while Python Numpy is the worse performing.

\begin{figure}[ht]
	\centering
	\includegraphics[scale=0.31]{memory-bar}
	\caption[Peak Memory Utilisation]{Peak Memory Utilized during training with different model sizes remain similar within the same implementation}
\end{figure}

\subsection[Python - Numpy]{Python Numpy based HDR-NN}

The Numpy implementation consistantly took longer duration to perform the same training cycle as compared to the C implementation

\subsection[Tensorflow Lite]{Tensorflow-Lite based HDR-NN}
\textit{Benchmark pending \dots}

\subsection[C]{C based HDR-NN}

C implementation had lower execution times and memory usage

\subsection[CPP - Eigen]{CPP based HDR-NN}
\textit{Benchmark pending \dots}

\section{CMSIS-NN based Optimisations to Training}
\textit{Further breakdown of the performance achieved from different optimisation techniques}

\subsection{Quantisation}
\textit{future: Training Network with Quantized weights}

\subsection{Pruning the Network}
\textit{future}

\section{Coefficient of variation}
A total of 10 iterations were conducted to ensure that the results remained consistent. To assess the degree of variability among the various trials, the mean and standard deviation were calculated across all runs, and their ratio was determined. This ratio indicates the level of variation between the different tests.

\chapter{Discussion}

\section{Developer Experience}

\section{Early stopping}
The training for all the implementations were executed by configuring the number of epochs as 30. This leads to the accuracy of model dropping significantly due to overfitting, which could be avoided if early stopping was implemented.
But, early stopping is not implemented as the performance would be completely different and there wouldn't be a standard setting to compare the implementations.

\chapter{Conclusion and Future Work}
\textit{What does it all mean? Where do we go from here?}


\backmatter
	% References
	% No restriction is set to the reference styles
	% Save your references in References.bib
	\nocite{*} % Remove this for your own citations
	\bibliographystyle{plain}
	\bibliography{References}

\end{document}